\documentclass[fontsize=11pt]{article}
\usepackage{amsmath}
\usepackage[utf8]{inputenc}
\usepackage[margin=0.75in]{geometry}

\title{CSC111 Project Proposal: Restaurant Recommendation System}
\author{Qingyi Jiang, Songxuan Wu, Rachel Yeung, Jiaqi Zhao}
\date{Monday, March 3, 2025}

\begin{document}
\maketitle

\section*{Problem Description and Research Question}

TODO

\section*{Computational Plan}

\subsection*{Use of Trees and Graphs}

\subsubsection*{Decision Tree for Filtering}
\begin{itemize}
    \item A decision tree is used to filter restaurants based on user preferences, such as cuisine type, budget range, and minimum rating.
    \item The tree structure follows a hierarchical filtering mechanism: 
    \begin{itemize}
        \item \textbf{Root Node}: Represents all available restaurants.
        \item \textbf{First Level}: Categorization based on cuisine type.
        \item \textbf{Second Level}: Further categorization based on budget.
        \item \textbf{Third Level}: Final filtering based on minimum rating.
        \item \textbf{Leaf Nodes}: Restaurants that meet all criteria.
    \end{itemize}
    \item This approach allows efficient filtering before applying ranking algorithms.
\end{itemize}

\subsubsection*{Graph-Based Ranking with PageRank}
\begin{itemize}
    \item A graph is constructed where: 
    \begin{itemize}
        \item \textbf{Nodes} represent individual restaurants.
        \item \textbf{Edges} connect restaurants with similar attributes (cuisine, ratings, and pricing).
        \item \textbf{Edge Weights} are computed based on similarity scores derived from restaurant attributes.
    \end{itemize}
    \item We apply PageRank to determine the relative importance of each restaurant in the network, ensuring that more "popular" restaurants (based on customer reviews and connectivity) are ranked higher.
\end{itemize}

\subsubsection*{Minimum Spanning Tree (MST) Optimization (Optional)}
\begin{itemize}
    \item As a refinement step, we use an MST-based approach to ensure that recommended restaurants maintain a degree of coherence.
    \item MST computation follows these steps: 
    \begin{enumerate}
        \item Construct a graph where nodes are restaurants, and edges are weighted by similarity.
        \item Compute the Minimum Spanning Tree (MST) to cluster related restaurants.
        \item Adjust PageRank results to reflect MST connectivity.
    \end{enumerate}
    \item This step ensures that recommendations are cohesive and logically structured, rather than isolated results.
\end{itemize}

\subsection*{Dataset and Sample Data}
Our project uses a real-world restaurant dataset containing the following key attributes:
\begin{itemize}
    \item \textbf{Name}: Restaurant name.
    \item \textbf{Address}: Physical location of the restaurant.
    \item \textbf{Online Order \& Booking Availability}: Whether the restaurant offers online orders and table booking.
    \item \textbf{Rating}: Customer rating (out of 5) based on reviews.
    \item \textbf{Restaurant Type}: Categorization such as casual dining, fine dining, buffet, etc.
    \item \textbf{Cuisines}: Types of cuisine served (e.g., North Indian, Chinese, Mughlai, Thai, etc.).
    \item \textbf{Approximate Cost}: Estimated cost for two people.
    \item \textbf{Dish Liked}: Popular dishes among customers.
\end{itemize}

\begin{tabular}{|l|l|l|l|l|}
    \hline
    \textbf{Name} & \textbf{Cuisines} & \textbf{Rating} & \textbf{Cost} & \textbf{Restaurant Type} \\
    \hline
    Jalsa & North Indian, Mughlai & 4.1 & 800 & Casual Dining \\
    \hline
    Spice Elephant & Chinese, North Indian, Thai & 4.1 & 800 & Casual Dining \\
    \hline
\end{tabular}

\subsection*{Types of Computation}
\begin{enumerate}
    \item \textbf{Data Preprocessing}
    \begin{itemize}
        \item Handle missing values and standardize categorical fields.
        \item Convert cuisine types into numerical encodings where needed.
        \item Normalize numerical attributes (ratings, votes, pricing).
    \end{itemize}
    \item \textbf{Decision Tree Filtering}
    \begin{itemize}
        \item Filter restaurants based on Cuisine Type $\rightarrow$ Budget $\rightarrow$ Minimum Rating.
    \end{itemize}
    \item \textbf{Graph-Based Ranking with PageRank}
    \begin{itemize}
        \item Construct a graph representation of the filtered restaurants.
        \item Compute restaurant similarity using cuisine, ratings, and cost.
        \item Apply PageRank to rank restaurants based on their connectivity in the similarity graph.
    \end{itemize}
    \item \textbf{(Optional) MST-Based Refinement}
    \begin{itemize}
        \item Construct an MST of the restaurant similarity graph.
        \item Adjust the PageRank ranking to prioritize more connected restaurants.
    \end{itemize}
\end{enumerate}

\subsection*{Use of Python Libraries}
\subsubsection*{pandas (Data Processing)}
\begin{itemize}
    \item \textbf{Usage}: Load, clean, and manipulate restaurant data.
    \item \textbf{Key Functions}: pandas.read\_csv(), pandas.DataFrame.query(), pandas.DataFrame.sort\_values()
\end{itemize}

\subsubsection*{networkx (Graph Construction \& Ranking Computation)}
\begin{itemize}
    \item \textbf{Usage}: Construct a graph representation of restaurants.
    \item \textbf{Key Functions}: networkx.Graph(), networkx.pagerank(), networkx.minimum\_spanning\_tree()
\end{itemize}

\subsubsection*{plotly (Visualization)}
\begin{itemize}
    \item \textbf{Usage}: Display restaurant rankings using bar charts.
    \item \textbf{Key Functions}: plotly.express.bar(), plotly.graph\_objects.Figure.update\_layout()
\end{itemize}

\subsection*{Expected Output and Visualization}
\begin{itemize}
    \item Basic Table Display: Restaurants ranked by PageRank score.
    \item Ranking Visualization: Bar charts using Plotly.
    \item Interactive Filtering (Optional): Users adjust preferences dynamically via Streamlit.
    \item Graph Visualization (Optional): Interactive graph representation of restaurant connections.
\end{itemize}

\section*{References}

TODO

\end{document}
